\documentclass[]{article}
\usepackage{lmodern}
\usepackage{amssymb,amsmath}
\usepackage{ifxetex,ifluatex}
\usepackage{fixltx2e} % provides \textsubscript
\ifnum 0\ifxetex 1\fi\ifluatex 1\fi=0 % if pdftex
  \usepackage[T1]{fontenc}
  \usepackage[utf8]{inputenc}
\else % if luatex or xelatex
  \ifxetex
    \usepackage{mathspec}
  \else
    \usepackage{fontspec}
  \fi
  \defaultfontfeatures{Ligatures=TeX,Scale=MatchLowercase}
\fi
% use upquote if available, for straight quotes in verbatim environments
\IfFileExists{upquote.sty}{\usepackage{upquote}}{}
% use microtype if available
\IfFileExists{microtype.sty}{%
\usepackage{microtype}
\UseMicrotypeSet[protrusion]{basicmath} % disable protrusion for tt fonts
}{}
\usepackage[margin=1in]{geometry}
\usepackage{hyperref}
\hypersetup{unicode=true,
            pdftitle={Assignment1},
            pdfauthor={Kelsie Ferin and Theo},
            pdfborder={0 0 0},
            breaklinks=true}
\urlstyle{same}  % don't use monospace font for urls
\usepackage{color}
\usepackage{fancyvrb}
\newcommand{\VerbBar}{|}
\newcommand{\VERB}{\Verb[commandchars=\\\{\}]}
\DefineVerbatimEnvironment{Highlighting}{Verbatim}{commandchars=\\\{\}}
% Add ',fontsize=\small' for more characters per line
\usepackage{framed}
\definecolor{shadecolor}{RGB}{248,248,248}
\newenvironment{Shaded}{\begin{snugshade}}{\end{snugshade}}
\newcommand{\KeywordTok}[1]{\textcolor[rgb]{0.13,0.29,0.53}{\textbf{#1}}}
\newcommand{\DataTypeTok}[1]{\textcolor[rgb]{0.13,0.29,0.53}{#1}}
\newcommand{\DecValTok}[1]{\textcolor[rgb]{0.00,0.00,0.81}{#1}}
\newcommand{\BaseNTok}[1]{\textcolor[rgb]{0.00,0.00,0.81}{#1}}
\newcommand{\FloatTok}[1]{\textcolor[rgb]{0.00,0.00,0.81}{#1}}
\newcommand{\ConstantTok}[1]{\textcolor[rgb]{0.00,0.00,0.00}{#1}}
\newcommand{\CharTok}[1]{\textcolor[rgb]{0.31,0.60,0.02}{#1}}
\newcommand{\SpecialCharTok}[1]{\textcolor[rgb]{0.00,0.00,0.00}{#1}}
\newcommand{\StringTok}[1]{\textcolor[rgb]{0.31,0.60,0.02}{#1}}
\newcommand{\VerbatimStringTok}[1]{\textcolor[rgb]{0.31,0.60,0.02}{#1}}
\newcommand{\SpecialStringTok}[1]{\textcolor[rgb]{0.31,0.60,0.02}{#1}}
\newcommand{\ImportTok}[1]{#1}
\newcommand{\CommentTok}[1]{\textcolor[rgb]{0.56,0.35,0.01}{\textit{#1}}}
\newcommand{\DocumentationTok}[1]{\textcolor[rgb]{0.56,0.35,0.01}{\textbf{\textit{#1}}}}
\newcommand{\AnnotationTok}[1]{\textcolor[rgb]{0.56,0.35,0.01}{\textbf{\textit{#1}}}}
\newcommand{\CommentVarTok}[1]{\textcolor[rgb]{0.56,0.35,0.01}{\textbf{\textit{#1}}}}
\newcommand{\OtherTok}[1]{\textcolor[rgb]{0.56,0.35,0.01}{#1}}
\newcommand{\FunctionTok}[1]{\textcolor[rgb]{0.00,0.00,0.00}{#1}}
\newcommand{\VariableTok}[1]{\textcolor[rgb]{0.00,0.00,0.00}{#1}}
\newcommand{\ControlFlowTok}[1]{\textcolor[rgb]{0.13,0.29,0.53}{\textbf{#1}}}
\newcommand{\OperatorTok}[1]{\textcolor[rgb]{0.81,0.36,0.00}{\textbf{#1}}}
\newcommand{\BuiltInTok}[1]{#1}
\newcommand{\ExtensionTok}[1]{#1}
\newcommand{\PreprocessorTok}[1]{\textcolor[rgb]{0.56,0.35,0.01}{\textit{#1}}}
\newcommand{\AttributeTok}[1]{\textcolor[rgb]{0.77,0.63,0.00}{#1}}
\newcommand{\RegionMarkerTok}[1]{#1}
\newcommand{\InformationTok}[1]{\textcolor[rgb]{0.56,0.35,0.01}{\textbf{\textit{#1}}}}
\newcommand{\WarningTok}[1]{\textcolor[rgb]{0.56,0.35,0.01}{\textbf{\textit{#1}}}}
\newcommand{\AlertTok}[1]{\textcolor[rgb]{0.94,0.16,0.16}{#1}}
\newcommand{\ErrorTok}[1]{\textcolor[rgb]{0.64,0.00,0.00}{\textbf{#1}}}
\newcommand{\NormalTok}[1]{#1}
\usepackage{graphicx,grffile}
\makeatletter
\def\maxwidth{\ifdim\Gin@nat@width>\linewidth\linewidth\else\Gin@nat@width\fi}
\def\maxheight{\ifdim\Gin@nat@height>\textheight\textheight\else\Gin@nat@height\fi}
\makeatother
% Scale images if necessary, so that they will not overflow the page
% margins by default, and it is still possible to overwrite the defaults
% using explicit options in \includegraphics[width, height, ...]{}
\setkeys{Gin}{width=\maxwidth,height=\maxheight,keepaspectratio}
\IfFileExists{parskip.sty}{%
\usepackage{parskip}
}{% else
\setlength{\parindent}{0pt}
\setlength{\parskip}{6pt plus 2pt minus 1pt}
}
\setlength{\emergencystretch}{3em}  % prevent overfull lines
\providecommand{\tightlist}{%
  \setlength{\itemsep}{0pt}\setlength{\parskip}{0pt}}
\setcounter{secnumdepth}{0}
% Redefines (sub)paragraphs to behave more like sections
\ifx\paragraph\undefined\else
\let\oldparagraph\paragraph
\renewcommand{\paragraph}[1]{\oldparagraph{#1}\mbox{}}
\fi
\ifx\subparagraph\undefined\else
\let\oldsubparagraph\subparagraph
\renewcommand{\subparagraph}[1]{\oldsubparagraph{#1}\mbox{}}
\fi

%%% Use protect on footnotes to avoid problems with footnotes in titles
\let\rmarkdownfootnote\footnote%
\def\footnote{\protect\rmarkdownfootnote}

%%% Change title format to be more compact
\usepackage{titling}

% Create subtitle command for use in maketitle
\newcommand{\subtitle}[1]{
  \posttitle{
    \begin{center}\large#1\end{center}
    }
}

\setlength{\droptitle}{-2em}
  \title{Assignment1}
  \pretitle{\vspace{\droptitle}\centering\huge}
  \posttitle{\par}
  \author{Kelsie Ferin and Theo}
  \preauthor{\centering\large\emph}
  \postauthor{\par}
  \predate{\centering\large\emph}
  \postdate{\par}
  \date{January 29, 2018}


\begin{document}
\maketitle

\subsection{Assignment 1}\label{assignment-1}

\subsubsection{Part 1: Example problem from the physics text book (pg.
105)}\label{part-1-example-problem-from-the-physics-text-book-pg.-105}

Below are the constants that will be used throuout the first part of the
code.

\begin{Shaded}
\begin{Highlighting}[]
\CommentTok{# Planck's constant [Js]}
\NormalTok{h=}\FloatTok{6.62606896e-34}
\CommentTok{# speed of light [m/s]}
\NormalTok{c=}\FloatTok{2.99792458e8} 
\CommentTok{# Boltzman constant [J/K]}
\NormalTok{k=}\FloatTok{1.3806504e-23} 

\CommentTok{# temperature used for sun in example [K]}
\NormalTok{T =}\StringTok{ }\DecValTok{5800}
\end{Highlighting}
\end{Shaded}

To calculate the following equations of a blackbody with a different
waveband, change sequence range for lambda!

\begin{Shaded}
\begin{Highlighting}[]
\CommentTok{# the following lambda will not be used, this is for the entire spectrum of visible light}
\CommentTok{#lambda = seq(3.4e-7,8.05e-7, by=1e-10)}

\CommentTok{# this lambda sequence is for the band of wavelengths used in the example problem [m]}
\NormalTok{lambda =}\StringTok{ }\KeywordTok{seq}\NormalTok{(}\FloatTok{6.0e-7}\NormalTok{,}\FloatTok{6.05e-7}\NormalTok{, }\DataTypeTok{by=}\FloatTok{1e-10}\NormalTok{)}
\end{Highlighting}
\end{Shaded}

The following 3 equations are for each 3 steps in the example problem.

Equation 1: The exponential portion of Planck's equation. This is
unitless.

\begin{Shaded}
\begin{Highlighting}[]
\CommentTok{# Equation 1}
\NormalTok{y =}\StringTok{ }\NormalTok{(h}\OperatorTok{*}\NormalTok{c)}\OperatorTok{/}\NormalTok{(lambda}\OperatorTok{*}\NormalTok{k}\OperatorTok{*}\NormalTok{T)}
\KeywordTok{plot}\NormalTok{(lambda,y,}\DataTypeTok{xlab=}\StringTok{'Wavelength (m)'}\NormalTok{,}\DataTypeTok{ylab=}\StringTok{'Plancks Exponential'}\NormalTok{,}\DataTypeTok{main=}\StringTok{'Part #1: Equation 1'}\NormalTok{)}
\end{Highlighting}
\end{Shaded}

\includegraphics{Assignment1_Ferin_Hartman_files/figure-latex/unnamed-chunk-3-1.pdf}

Equation \#2: Spectral Emittance of the sun (I; W/m\^{}2/m)

\begin{Shaded}
\begin{Highlighting}[]
\NormalTok{I=((}\DecValTok{2}\OperatorTok{*}\NormalTok{pi)}\OperatorTok{*}\NormalTok{(h)}\OperatorTok{*}\NormalTok{(c}\OperatorTok{**}\DecValTok{2}\NormalTok{)}\OperatorTok{/}\NormalTok{((lambda}\OperatorTok{**}\DecValTok{5}\NormalTok{)}\OperatorTok{*}\NormalTok{(}\KeywordTok{exp}\NormalTok{(y)}\OperatorTok{-}\DecValTok{1}\NormalTok{)))}
\KeywordTok{plot}\NormalTok{(lambda,I,}\DataTypeTok{xlab=}\StringTok{'Wavelength (m)'}\NormalTok{,}\DataTypeTok{ylab=}\StringTok{'Spectral Emittance (W/m^2/m)'}\NormalTok{,}\DataTypeTok{main=}\StringTok{'Part #1: Equation 2'}\NormalTok{)}
\end{Highlighting}
\end{Shaded}

\includegraphics{Assignment1_Ferin_Hartman_files/figure-latex/unnamed-chunk-4-1.pdf}

Equation \#3: This plot is the intesnity (MW/m\^{}2) for the given
spectral emittance (W/m\^{}2/m) values calculated in equation 2.

\begin{Shaded}
\begin{Highlighting}[]
\NormalTok{Ilambda =}\StringTok{ }\NormalTok{(I}\OperatorTok{*}\NormalTok{(}\KeywordTok{max}\NormalTok{(lambda)}\OperatorTok{-}\KeywordTok{min}\NormalTok{(lambda)))}\OperatorTok{*}\FloatTok{1e-6}
\KeywordTok{plot}\NormalTok{(I,Ilambda,}\DataTypeTok{xlab=}\StringTok{'Spectral Emittance (W/m^2/m)'}\NormalTok{,}\DataTypeTok{ylab=}\StringTok{'Intensity (MW/m^2)'}\NormalTok{,}\DataTypeTok{main=}\StringTok{'Part #1: Equation 3'}\NormalTok{)}
\end{Highlighting}
\end{Shaded}

\includegraphics{Assignment1_Ferin_Hartman_files/figure-latex/unnamed-chunk-5-1.pdf}

\subsubsection{Part 2: Power within a
waveband}\label{part-2-power-within-a-waveband}

Here we will compare the total amount of power of the blackbody
calculation to the data collected from the spectral radiometer in class.
Just as a reminder, spectral emittance is what Plancks's Law gives for
the sun with this temperature. Spectral irradience is what is measured
here on Earth. These values will NOT be the same due to Earth's
atmosphere and its absorbtion in some wavelengths

The equations in part 1 use {[}m{]} for the wavelengths. We need to
convert these to nm to stay consistent with the collected data.

\begin{Shaded}
\begin{Highlighting}[]
\NormalTok{lambda_nm=lambda}\OperatorTok{*}\FloatTok{1e+9}
\NormalTok{I_new =}\StringTok{ }\NormalTok{I}\OperatorTok{/}\FloatTok{1e+9}
\KeywordTok{plot}\NormalTok{(lambda_nm,I_new,}\DataTypeTok{xlab=}\StringTok{'Wavelength (nm)'}\NormalTok{,}\DataTypeTok{ylab=}\StringTok{'Spectral Emittance (W/m^2/nm)'}\NormalTok{,}\DataTypeTok{main=}\StringTok{'Part #2: Equation 1'}\NormalTok{)}
\end{Highlighting}
\end{Shaded}

\includegraphics{Assignment1_Ferin_Hartman_files/figure-latex/unnamed-chunk-6-1.pdf}

Now we need to use the values collected during class. The following
chunk of code is from Anabelle:

\begin{Shaded}
\begin{Highlighting}[]
\CommentTok{# Some cleaning}
\NormalTok{data<-tab[}\DecValTok{2}\OperatorTok{:}\DecValTok{482}\NormalTok{,] }\CommentTok{# select rows}
\NormalTok{data<-data }\OperatorTok\StringTok{ }\KeywordTok{rename}\NormalTok{(}\DataTypeTok{waveL=}\NormalTok{Timestamp) }\CommentTok{# rename column }
\NormalTok{data}\OperatorTok{$}\NormalTok{waveL<-}\KeywordTok{as.numeric}\NormalTok{(}\KeywordTok{as.vector}\NormalTok{(data}\OperatorTok{$}\NormalTok{waveL)) }\CommentTok{# to get a x_continous scale, data need to be numeric not factor}

\CommentTok{#Plot the Spectral Irradiance as a function of wavelength (nm)}
\NormalTok{time<-data[,}\DecValTok{82}\NormalTok{]}\CommentTok{# here I just pick a column randomly}
\CommentTok{#class(time)}
\CommentTok{# time has to be converted into numeric }
\NormalTok{time<-}\KeywordTok{as.numeric}\NormalTok{(}\KeywordTok{unlist}\NormalTok{(time)) }\CommentTok{# I used unlist to convert all the element to a single numeric vector}
\KeywordTok{ggplot}\NormalTok{(}\DataTypeTok{data=}\NormalTok{data,}\KeywordTok{aes}\NormalTok{(}\DataTypeTok{x=}\NormalTok{waveL,}\DataTypeTok{y=}\NormalTok{time))}\OperatorTok{+}
\StringTok{  }\KeywordTok{geom_line}\NormalTok{() }\OperatorTok{+}
\StringTok{  }\KeywordTok{scale_x_continuous}\NormalTok{(}\DataTypeTok{breaks =} \KeywordTok{round}\NormalTok{(}\KeywordTok{seq}\NormalTok{(}\KeywordTok{min}\NormalTok{(data}\OperatorTok{$}\NormalTok{waveL), }\KeywordTok{max}\NormalTok{(data}\OperatorTok{$}\NormalTok{waveL), }\DataTypeTok{by =} \DecValTok{100}\NormalTok{),}\DecValTok{1}\NormalTok{)) }\OperatorTok{+}
\StringTok{  }\KeywordTok{scale_y_continuous}\NormalTok{(}\DataTypeTok{breaks =} \KeywordTok{round}\NormalTok{(}\KeywordTok{seq}\NormalTok{(}\KeywordTok{min}\NormalTok{(time), }\KeywordTok{max}\NormalTok{(time), }\DataTypeTok{by =} \FloatTok{0.005}\NormalTok{),}\DecValTok{1}\NormalTok{)) }\OperatorTok{+}
\StringTok{  }\KeywordTok{xlab}\NormalTok{(}\StringTok{"wavelength (nm)"}\NormalTok{) }\OperatorTok{+}
\StringTok{  }\KeywordTok{ylab}\NormalTok{(}\StringTok{"Spectral Irradiance (W/m2/micrometer)"}\NormalTok{)}
\end{Highlighting}
\end{Shaded}

\includegraphics{Assignment1_Ferin_Hartman_files/figure-latex/unnamed-chunk-8-1.pdf}

First we need to convert the spectral irradiance from W/m2/micrometer to
W/m2/nm:

\begin{Shaded}
\begin{Highlighting}[]
\NormalTok{irradiance =}\StringTok{ }\NormalTok{time}\OperatorTok{*}\FloatTok{1e+3}
\NormalTok{wavelength =}\StringTok{ }\NormalTok{data}\OperatorTok{$}\NormalTok{waveL}
\KeywordTok{plot}\NormalTok{(wavelength,irradiance,}\DataTypeTok{type=}\StringTok{"o"}\NormalTok{,}\DataTypeTok{xlab=}\StringTok{'Wavelength (nm)'}\NormalTok{,}\DataTypeTok{ylab=}\StringTok{'Spectral Irradiance (W/m^2/nm)'}\NormalTok{,}\DataTypeTok{main=}\StringTok{'Part #2: Collected Data'}\NormalTok{)}
\end{Highlighting}
\end{Shaded}

\includegraphics{Assignment1_Ferin_Hartman_files/figure-latex/unnamed-chunk-9-1.pdf}

We only want to use the wavelength band of 600-605nm to compare to the
black body, which is shown in the following plot.

\begin{Shaded}
\begin{Highlighting}[]
\NormalTok{irradiance_data =}\StringTok{ }\NormalTok{irradiance[}\DecValTok{261}\OperatorTok{:}\DecValTok{266}\NormalTok{]}
\NormalTok{band_data =}\StringTok{ }\NormalTok{data}\OperatorTok{$}\NormalTok{waveL[}\DecValTok{261}\OperatorTok{:}\DecValTok{266}\NormalTok{]}
\KeywordTok{plot}\NormalTok{(band_data,irradiance_data,}\DataTypeTok{type=}\StringTok{"o"}\NormalTok{,}\DataTypeTok{xlab=}\StringTok{'Wavelength (nm)'}\NormalTok{,}\DataTypeTok{ylab=}\StringTok{'Spectral Irradiance (W/m^2/nm)'}\NormalTok{,}\DataTypeTok{main=}\StringTok{'Part #2: Collected Data'}\NormalTok{)}
\end{Highlighting}
\end{Shaded}

\includegraphics{Assignment1_Ferin_Hartman_files/figure-latex/unnamed-chunk-10-1.pdf}

Lastly, we need to integrate the area under the curve for both the
blackbody and observed values:

\begin{Shaded}
\begin{Highlighting}[]
\CommentTok{# Blackbody Integration (emittance)}
\NormalTok{bb_E_integ =}\StringTok{ }\KeywordTok{sum}\NormalTok{(I_new }\OperatorTok{*}\StringTok{ }\FloatTok{0.1}\NormalTok{)}
\KeywordTok{cat}\NormalTok{(}\StringTok{"Blackbody value of total power: "}\NormalTok{, bb_E_integ, }\StringTok{"W/m^2"}\NormalTok{)}
\end{Highlighting}
\end{Shaded}

\begin{verbatim}
## Blackbody value of total power:  398006.3 W/m^2
\end{verbatim}

\begin{Shaded}
\begin{Highlighting}[]
\CommentTok{# Observed Integration (irradiance)}
\NormalTok{obs_I_integ =}\StringTok{ }\KeywordTok{sum}\NormalTok{(irradiance_data)}
\KeywordTok{cat}\NormalTok{(}\StringTok{"Observed value of total power: "}\NormalTok{, obs_I_integ, }\StringTok{"W/m^2"}\NormalTok{)}
\end{Highlighting}
\end{Shaded}

\begin{verbatim}
## Observed value of total power:  4582.41 W/m^2
\end{verbatim}

\subsubsection{Part 3: Example problem; Photons required for
sight!}\label{part-3-example-problem-photons-required-for-sight}

First, we define the constants that will be used in these calculations:

\begin{Shaded}
\begin{Highlighting}[]
\CommentTok{#Intensity of radiation reaching the eye in order for the human eye to percieve light.}
\NormalTok{I =}\StringTok{ }\FloatTok{4.0e-11} \CommentTok{# Units - W/m^2}

\CommentTok{#Plancks Constant }
\NormalTok{h =}\StringTok{ }\FloatTok{6.63e-34} \CommentTok{# Units - J*s}

\CommentTok{#Speed of light in a vacuum}
\NormalTok{c =}\StringTok{ }\FloatTok{3.0e8} \CommentTok{# Units - m/s}
\end{Highlighting}
\end{Shaded}

Secondly, we calculate the area of the pupil: (According to NCBI the
diameter of a human pupil ranges from 2 mm to 8mm)

\begin{Shaded}
\begin{Highlighting}[]
\CommentTok{#Diameter of a human pupil}
\NormalTok{d =}\StringTok{ }\KeywordTok{seq}\NormalTok{(}\DecValTok{2}\NormalTok{,}\DecValTok{8}\NormalTok{,}\DataTypeTok{by =} \DecValTok{1}\NormalTok{) }\CommentTok{# Units - mm}

\CommentTok{#Convert the diameter to meters}
\NormalTok{d =}\StringTok{ }\NormalTok{d}\OperatorTok{*}\FloatTok{1e-3} \CommentTok{# Units - m }

\CommentTok{#Calculate the pupil area}
\NormalTok{pupil_area =}\StringTok{ }\NormalTok{pi}\OperatorTok{*}\NormalTok{((d}\OperatorTok{/}\DecValTok{2}\NormalTok{)}\OperatorTok{^}\DecValTok{2}\NormalTok{) }\CommentTok{# Units - m^2}
\end{Highlighting}
\end{Shaded}

Third, we calculate the power reaching the eye:

\begin{Shaded}
\begin{Highlighting}[]
\CommentTok{#Power reaching the eye}
\NormalTok{Power_reach_eye =}\StringTok{ }\NormalTok{I}\OperatorTok{*}\NormalTok{pupil_area }\CommentTok{# Units - W}
\end{Highlighting}
\end{Shaded}

Fourth, we calculate the energy of a photon for wavelengths in the
visible spectrum:

\begin{Shaded}
\begin{Highlighting}[]
\CommentTok{#Range of wavelength in the visible spectrum}
\NormalTok{lam_pupil =}\StringTok{ }\KeywordTok{seq}\NormalTok{(}\FloatTok{380e-9}\NormalTok{,}\FloatTok{780e-9}\NormalTok{,}\DataTypeTok{by =} \FloatTok{1e-9}\NormalTok{) }\CommentTok{#Units - m}

\CommentTok{#Energy of a photon}
\NormalTok{e =}\StringTok{ }\NormalTok{(h}\OperatorTok{*}\NormalTok{c)}\OperatorTok{/}\NormalTok{lam_pupil }\CommentTok{#Units - J}
\end{Highlighting}
\end{Shaded}

Fifth, we calculate the amount of photons reaching the eye

\begin{Shaded}
\begin{Highlighting}[]
\CommentTok{#Create Vector for Calculations}
\NormalTok{Photons_reach_eye =}\StringTok{ }\KeywordTok{matrix}\NormalTok{(}\DecValTok{0}\NormalTok{,}\DecValTok{401}\NormalTok{,}\DecValTok{7}\NormalTok{)}

\CommentTok{#Number of photons reaching the eye}
\ControlFlowTok{for}\NormalTok{(i }\ControlFlowTok{in} \DecValTok{1}\OperatorTok{:}\DecValTok{401}\NormalTok{)\{}
  \ControlFlowTok{for}\NormalTok{(j }\ControlFlowTok{in} \DecValTok{1}\OperatorTok{:}\DecValTok{7}\NormalTok{)\{}
\NormalTok{    Photons_reach_eye[i,j] =}\StringTok{ }\NormalTok{Power_reach_eye[j]}\OperatorTok{/}\NormalTok{e[i] }\CommentTok{#Units - photons/s}
\NormalTok{  \}}
\NormalTok{\}}
\end{Highlighting}
\end{Shaded}

Sixth, we plot the results of this experiment:

\begin{Shaded}
\begin{Highlighting}[]
\KeywordTok{library}\NormalTok{(ggplot2)}
\CommentTok{#Set data up in a frame to plot}
\NormalTok{data_pup =}\StringTok{ }\KeywordTok{data.frame}\NormalTok{(}\DataTypeTok{x =}\NormalTok{ lam_pupil,}\DataTypeTok{y1 =}\NormalTok{ Photons_reach_eye[,}\DecValTok{1}\NormalTok{], }\DataTypeTok{y2 =}\NormalTok{ Photons_reach_eye[,}\DecValTok{2}\NormalTok{], }\DataTypeTok{y3 =}\NormalTok{ Photons_reach_eye[,}\DecValTok{3}\NormalTok{], }\DataTypeTok{y4 =}\NormalTok{ Photons_reach_eye[,}\DecValTok{4}\NormalTok{], }\DataTypeTok{y5 =}\NormalTok{ Photons_reach_eye[,}\DecValTok{5}\NormalTok{], }\DataTypeTok{y6 =}\NormalTok{ Photons_reach_eye[,}\DecValTok{6}\NormalTok{], }\DataTypeTok{y7 =}\NormalTok{ Photons_reach_eye[,}\DecValTok{7}\NormalTok{])}

\CommentTok{#Plotting }
\NormalTok{col_set =}\StringTok{ }\KeywordTok{rainbow}\NormalTok{(}\KeywordTok{ncol}\NormalTok{(Photons_reach_eye))}
\KeywordTok{matplot}\NormalTok{(lam_pupil,Photons_reach_eye,}\DataTypeTok{type =} \StringTok{"l"}\NormalTok{,}\DataTypeTok{main =} \StringTok{"Photons That Reach The Human Eye at Different Wavelengths with Different Pupil Sizes"}\NormalTok{,}\DataTypeTok{cex.main =}\NormalTok{ .}\DecValTok{8}\NormalTok{,}\DataTypeTok{xlab =} \StringTok{"Wavelength (m)"}\NormalTok{,}\DataTypeTok{ylab =} \StringTok{"Number of Photons"}\NormalTok{,}\DataTypeTok{lty =} \StringTok{"solid"}\NormalTok{, }\DataTypeTok{lwd =} \StringTok{"2"}\NormalTok{,}\DataTypeTok{col =}\NormalTok{ col_set)}
\NormalTok{leg.txt <-}\StringTok{ }\KeywordTok{c}\NormalTok{(}\StringTok{"Pupil Size = 2mm"}\NormalTok{,}\StringTok{"Pupil Size = 3mm"}\NormalTok{,}\StringTok{"Pupil Size = 4mm"}\NormalTok{,}\StringTok{"Pupil Size = 5mm"}\NormalTok{,}\StringTok{"Pupil Size = 6mm"}\NormalTok{,}\StringTok{"Pupil Size = 7mm"}\NormalTok{,}\StringTok{"Pupil Size = 8mm"}\NormalTok{)}
\NormalTok{cex =}\StringTok{ }\NormalTok{.}\DecValTok{5}
\KeywordTok{legend}\NormalTok{(lam_pupil[}\DecValTok{1}\NormalTok{],Photons_reach_eye[}\DecValTok{401}\NormalTok{,}\DecValTok{7}\NormalTok{],leg.txt,}\DataTypeTok{cex =}\NormalTok{ .}\DecValTok{5}\NormalTok{,}\DataTypeTok{lwd =} \DecValTok{2}\NormalTok{, }\DataTypeTok{col =}\NormalTok{ col_set)}
\end{Highlighting}
\end{Shaded}

\includegraphics{Assignment1_Ferin_Hartman_files/figure-latex/unnamed-chunk-18-1.pdf}

\subsection{Now Lets Use The Spectrum Which We Measured To See How Many
Photons Our Eyes Were
Intercepting}\label{now-lets-use-the-spectrum-which-we-measured-to-see-how-many-photons-our-eyes-were-intercepting}

\begin{Shaded}
\begin{Highlighting}[]
\CommentTok{#Because we can calculate the number of photons by taking the Intensity of the Radiation in our pupil divided by the energy of a photon at that wavelength, we can take the intensity measured divided by the power at that wavelength and come up with a total number of photons!}

\CommentTok{#Lets assume a pupil size of 6mm}
\CommentTok{#Calculating the Intensity of radiation hitting your eye}
\NormalTok{Intensity_pupil =}\StringTok{ }\NormalTok{time }\OperatorTok{*}\StringTok{ }\NormalTok{pupil_area[}\DecValTok{5}\NormalTok{] }\CommentTok{#Units - W}

\CommentTok{#Calculating the photon energy of the wavelengths that we measured}

\NormalTok{e_meas =}\StringTok{ }\NormalTok{(h}\OperatorTok{*}\NormalTok{c)}\OperatorTok{/}\NormalTok{(wavelength}\OperatorTok{*}\FloatTok{1e-9}\NormalTok{) }\CommentTok{#Units - J}

\CommentTok{#Now to calculate the number of photons that are making it to our eyes}

\NormalTok{Photons_reach_eye_measured =}\StringTok{ }\NormalTok{Intensity_pupil}\OperatorTok{/}\NormalTok{e_meas}\CommentTok{#Units - Photons/s}

\CommentTok{#If we sum up all of the photons from every wavelength, we see about how many photons per second are hitting our eyes}

\NormalTok{Total_Photons =}\StringTok{ }\KeywordTok{sum}\NormalTok{(Photons_reach_eye_measured)}\CommentTok{#Units - Photons/s}
\KeywordTok{cat}\NormalTok{(}\StringTok{"Total Photons Reaching Our Eyes With Our Measured Data:"}\NormalTok{, Total_Photons,}\StringTok{" Photons/s"}\NormalTok{)}
\end{Highlighting}
\end{Shaded}

\begin{verbatim}
## Total Photons Reaching Our Eyes With Our Measured Data: 3.215567e+16  Photons/s
\end{verbatim}


\end{document}
